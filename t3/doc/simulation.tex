\newpage
\section{Simulation Analysis}
\label{sec:simulation}
Since this circuit has a sinusoidal voltage source, the voltage and current values of the various components vary in time. Therefore, we must perform a transient analysis to simulate the circuit's total response. We also ran operating point analysis for both $t<0$ and $t=0$ to determine what the initial conditions were and establish the boundary conditions. Finally, we ran a frequency response analysis of the capacitor voltage and node 6 voltage.
\subsection{Operating Point Analysis}
The tables below show the simulated operating point results for both $t<0$ (where we assume no current is flowing through the capacitor) and t=0 where we shut down the voltage source $V_{s}$ and we replace the capacitor with a voltage source $V_{x}$, in order to determine the equivalent resistor seen through the capacitor.


Table shows the simulated node voltages and branch currents for $t<0$. Table ~\ref{tab:2} gives us the current flowing through our voltage source, $V_{x}$. With the voltage source value calculated in the previous operating poing simulation, we can now calculate the equivalent resistance, $R_{eq}$, and characteristic time, $\tau$:

\begin{gather*}
  R_{eq} = \frac{V_x}{@V_c[i]} \Leftrightarrow R_{eq} = \frac{8.548722}{2.80822e-03} \Leftrightarrow R_{eq} = 3044.178163 Ohm
\end{gather*}

\begin{gather*}
  \tau = R_{eq} \times C \Leftrightarrow \tau = (3044.178163*C)   s
\end{gather*}

\newpage
\subsection{Natural response}
In this section we simulate the natural response of the circuit in the [0,20] ms time interval using the transient analysis simulation in NGSpice and the computations from the previous section.
\subsection{Natural and forced response}
We repeat the previous step now with the sinusoidal voltage source $V_{s}(t)$ considering a frequency of 1kHz. Below is the plot for both the stimulus and the response:
\newpage
\subsection{Frequency response}
We simulated the frequency response to the voltage source $V_{s}$, both in node 6 and in the capacitor itself, in NGSpice, and we plotted the magnitude, in dB, and phase, in degrees, of the three voltages in relation of course to a variation of frequency. The frequency varies from 0.1 Hz to 1MHz. Analysing figures ~\ref{fig:magnitude} and ~\ref{fig:phase}, they confirm what we foresaw in the theoretical analysis. The various reasons on why certain voltage parameters vary or are equivalent are laid out and explained in the $Frequency$ $Response$ subsection of the theoretical analysis.
\par
