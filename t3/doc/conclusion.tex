\newpage
\section{Conclusion}
\label{sec:conclusion}

Now, we proceed to compare our theoretical and simulation results. Looking at all plots generated either by GNU Octave or NGSpice, the theoretical plots are checked out by NGspice with very high precision (plots relative to the natural solution, the complete solution and the frequency responses). When it comes to the tables generated by the operating point analysis regarding $t<0$ and $t=0$, we have assembled a table comparing the theoretical analysis run by Octave and the simulated data from NGSpice (Table ~\ref{tab:comp}).
\begin{table}[!h]
  \centering
  \begin{tabular}{c c c c}
    \hline    
    {\bf Theoretical} & {\bf Value} & {\bf Simulation} & {\bf Value}\\ \hline
    @$I_{R1}$ & 0.219167 mA & @r1[i] & 2.191669e-04 A\\ \hline
@$I_{R2}$ & 0.229771 mA & @r2[i] & 2.297712e-04 A\\ \hline
@$I_{R3}$ & 0.010604 mA & @r3[i] & 1.060424e-05 A\\ \hline
@$I_{R4}$ & 1.185502 mA & @r4[i] & 1.185502e-03 A\\ \hline
@$I_{R5}$ & 0.229771 mA & @r5[i] & 2.297712e-04 A\\ \hline
@$I_{R6}$ & 0.966335 mA & @r6[i] & 9.663348e-04 A\\ \hline
@$I_{R7}$ & 0.966335 mA & @r7[i] & 9.663348e-04 A\\ \hline
$V_{1}$ & 5.113399 V &	v(1) & 5.113399e+00 V\\ \hline
$V_{2}$ & 4.889447 V &	v(2) & 4.889448e+00 V\\ \hline
$V_{3}$ & 4.428712 V &	v(3) & 4.428712e+00 V\\ \hline
$V_{5}$ & 4.921444 V &	v(5) & 4.921444e+00 V\\ \hline
$V_{6}$ & 5.620907 V &	v(6) & 5.620908e+00 V\\ \hline
$V_{7}$ & -1.953900 V & v(7) & -1.95390e+00 V\\ \hline
$V_{8}$ & -2.927814 V & v(8) & -2.92781e+00 V\\ \hline
-- & -- & -- & --\\ \hline
- & - &	@g[i] & -4.25263e-18 A\\ \hline
- & - &	@r1[i] & 4.056361e-18 A\\ \hline
- & - &	@r2[i] & 4.252625e-18 A\\ \hline
- & - &	@r3[i] & 1.962643e-19 A\\ \hline
- & - &	@r4[i] & -8.55795e-19 A\\ \hline
- & - &        @r5[i] & 2.808224e-03 A\\ \hline
- & - &	@r6[i] & -8.67362e-19 A\\ \hline
- & - &	@r7[i] & -1.78493e-18 A\\ \hline
- & - &	v(1) & 0.000000e+00 V\\ \hline
- & - &	v(2) & -4.14491e-15 V\\ \hline
- & - &	v(3) & -1.26722e-14 V\\ \hline
- & - &	v(5) & -3.55271e-15 V\\ \hline
$V_{6}$ & 8.548721 V & v(6) & 8.548722e+00 V\\ \hline
- & - &	v(7) & 1.753779e-15 V\\ \hline
$V_{8}$ & 0.000000 V & v(8) & 3.552714e-15 V\\ \hline
$V_{x}$ & 8.548721 V & vx & 8.548722e+00 V\\ \hline
@$I_{x}$ & 2.808224 mA & @vc[i] & 2.80822e-03 A\\ \hline
$R_{eq}$ & 3.044174 kOhm & $R_{eq}$ & 3044.178163 Ohm\\ \hline

 
  \end{tabular}
  \caption{Comparison of the theoretical and simulated data results, regarding $t<0$, and $t=0$ (initial conditions).}
  \label{tab:comp}
\end{table}

All compared figures are equal, if we discard effectuated roundings, mostly operated by Octave, and if we disregard differences in the number of significant algharisms presented by both softwares. In the final case of the equivalent resistance, $R_{eq}$, the error is bigger because we calculated it not using either of the softwares (we took the $V_{x}$ and $I_{x}$ data from NGspice and operated the division ourselves).
All comparisons made, all results predicted in the theoretical section were matched by the simulations executed.

All things taken into account, in this laboratory assignment, we were successful in producing coherent calculations using the Octave Maths tool and the circuit simulation, done using the Ngspice tool. Static and transient analysis were performed both theoretically and using a circuit simulation. Some results in the simulation that were predicted to yield zero current or voltage were actually very slightly off - in the magnitude of 1e-15, a similar magnitude to the floating number precision used in the software we utilised, which shows these errors are most likely due to roundings effectuated either by GNU Octave or NGSpice in their calculations, most specifically in linear systems. This could also be do to the writing and reading of the different text files truncating the numbers at specific decimal places. The whole report has been automatised and we're confident that it would yield consistent results with a new set of data.
