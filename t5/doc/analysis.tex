\section{Theoretical Analysis}
\label{sec:analysis}
\subsection{Circuit frequency response}
With the following equations we determined both cutoff frequencies for the band pass circuit. 

\begin{equation}
w_L=\frac{1}{R1C1}
\end{equation}
\begin{equation}
w_L=\frac{1}{R2C2}
\end{equation}
This was the definition used to determine the central frequency, which is meant to be 1KHz.
\begin{equation}
w_O=\sqrt{w_{H}w_{L}}
\end{equation}

\begin{table}[h!]
  \centering
  \begin{tabular}{|l|r|}
    \hline    
    {\bf Name} & {\bf Values} \\ \hline
    \input{frequency_tab} 
  \end{tabular}
  \caption{Cut off frequencies and central frequency}
  \label{tab:data}
\end{table}

\begin{figure}[!h] \centering
\includegraphics[width=0.6\linewidth]{gain.eps}
\caption{Voltage gain frequency response.}
\label{fig:gainfreq}
\end{figure}

\begin{figure}[!h] \centering
\includegraphics[width=0.6\linewidth]{phase.eps}
\caption{Voltage phase frequency response.}
\label{fig:gainfreq}
\end{figure}

\subsection{Central frequency results}
For the mentioned central frequency we computed the circuit gain and input and output impedances for the amplifier. Also, we determined a theoretical figure of merit based on the predicted results with the set of values we chose.
\begin{table}[h!]
  \centering
  \begin{tabular}{|l|r|}
    \hline    
    {\bf Name} & {\bf Values} \\ \hline
    \input{theo_tab} 
  \end{tabular}
  \caption{Voltage gain of the different stages and input/output impedances of the amplifier.}
  \label{tab:data}
\end{table}
