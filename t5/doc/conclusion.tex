\section{Conclusion}
\label{sec:conclusion}
\par
\begin{table}[!h]
  \centering
  \begin{tabular}{c c c c}
    \hline    
    {\bf Theoretical} & {\bf Value} & {\bf Simulation} & {\bf Value}\\ \hline
    @$I_{R1}$ & 0.219167 mA & @r1[i] & 2.191669e-04 A\\ \hline
@$I_{R2}$ & 0.229771 mA & @r2[i] & 2.297712e-04 A\\ \hline
@$I_{R3}$ & 0.010604 mA & @r3[i] & 1.060424e-05 A\\ \hline
@$I_{R4}$ & 1.185502 mA & @r4[i] & 1.185502e-03 A\\ \hline
@$I_{R5}$ & 0.229771 mA & @r5[i] & 2.297712e-04 A\\ \hline
@$I_{R6}$ & 0.966335 mA & @r6[i] & 9.663348e-04 A\\ \hline
@$I_{R7}$ & 0.966335 mA & @r7[i] & 9.663348e-04 A\\ \hline
$V_{1}$ & 5.113399 V &	v(1) & 5.113399e+00 V\\ \hline
$V_{2}$ & 4.889447 V &	v(2) & 4.889448e+00 V\\ \hline
$V_{3}$ & 4.428712 V &	v(3) & 4.428712e+00 V\\ \hline
$V_{5}$ & 4.921444 V &	v(5) & 4.921444e+00 V\\ \hline
$V_{6}$ & 5.620907 V &	v(6) & 5.620908e+00 V\\ \hline
$V_{7}$ & -1.953900 V & v(7) & -1.95390e+00 V\\ \hline
$V_{8}$ & -2.927814 V & v(8) & -2.92781e+00 V\\ \hline
-- & -- & -- & --\\ \hline
- & - &	@g[i] & -4.25263e-18 A\\ \hline
- & - &	@r1[i] & 4.056361e-18 A\\ \hline
- & - &	@r2[i] & 4.252625e-18 A\\ \hline
- & - &	@r3[i] & 1.962643e-19 A\\ \hline
- & - &	@r4[i] & -8.55795e-19 A\\ \hline
- & - &        @r5[i] & 2.808224e-03 A\\ \hline
- & - &	@r6[i] & -8.67362e-19 A\\ \hline
- & - &	@r7[i] & -1.78493e-18 A\\ \hline
- & - &	v(1) & 0.000000e+00 V\\ \hline
- & - &	v(2) & -4.14491e-15 V\\ \hline
- & - &	v(3) & -1.26722e-14 V\\ \hline
- & - &	v(5) & -3.55271e-15 V\\ \hline
$V_{6}$ & 8.548721 V & v(6) & 8.548722e+00 V\\ \hline
- & - &	v(7) & 1.753779e-15 V\\ \hline
$V_{8}$ & 0.000000 V & v(8) & 3.552714e-15 V\\ \hline
$V_{x}$ & 8.548721 V & vx & 8.548722e+00 V\\ \hline
@$I_{x}$ & 2.808224 mA & @vc[i] & 2.80822e-03 A\\ \hline
$R_{eq}$ & 3.044174 kOhm & $R_{eq}$ & 3044.178163 Ohm\\ \hline

 
  \end{tabular}
  \caption{Comparison of the theoretical and simulated data results, regarding the frequency response and impedances.}
  \label{tab:comp}
\end{table}

In this laboratory assignment, we managed to build a BandPass Filter (BPF) circuit which is represented in Figure 1. The first step of our analysis was to determine the frequency response by computing the transfer function of the whole circuit.

In the last report we explained how the theoretical model of the transistor lacked the complexity needed to yield closer results to the simulation. In this assignment we are dealing with 22 separate transistors inside the OP-AMP. This alone means that theoretical model used to analyse the circuit can differ significantly since it isn't expected to take into account the non-linearity of transistors. Adding to that is the complexity of the OP-AMP model used in Ngspice, especially the use of various capacitors and diodes which weren't taken into account in the octave analysis.

Due to this, the theoretical and simulation results differ significantly in some aspects. 
The bandwidth, despite still being approximately centered around the 1 KHz frequency, was much bigger (double the width).
%não sei se isto faz muito sentido

%This is due mainly (as would be expected by now) to the fact that the OPAMP model used includes capacitors, which introduces, by means of additional complex impedances, two extra poles in the transfer function. Not only does this subvert the format for the phase bode plot obtained theoretically, it also affects the upper cutoff frequency, and, as a result, the central passband frequency: while, in theory, the parameters used will not give us the desired frequency (in fact it would be around 100Hz and not around 1kHz), in practice, they do! \par

Despite all of this, the merit figures are somewhat similar, especially if we consider the significant differences from previous assignments.
