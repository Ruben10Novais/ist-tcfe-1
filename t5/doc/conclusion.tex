\section{Conclusion}
\label{sec:conclusion}
\par
\begin{table}[!h]
  \centering
  \begin{tabular}{c c c c}
    \hline    
    {\bf Theoretical} & {\bf Value} & {\bf Simulation} & {\bf Value}\\ \hline
    $V_{DCenvelope}$ & 25.255143 V & $V_{DCenvelope}$ & 2.387308e+01 V\\ \hline
$V_{ACenvelope}$ & 0.000311 V & $V_{ACenvelope}$ & 1.480000e-03 V\\ \hline
$V_{DCregulator}$ & 12.000000 V & $V_{DCregulator}$ & 1.200005e+01 V\\ \hline
$V_{ACregulator}$ & 15 uV & $V_{ACregulator}$ & 60 uV\\ \hline
$Merit$ & 33.573648 & $Merit$ & 4.735602e+00\\ \hline
 
  \end{tabular}
  \caption{Comparison of the theoretical and simulated data results, regarding the frequency response and impedances.}
  \label{tab:comp}
\end{table}

In this laboratory assignment, we managed to build a BandPass Filter (BPF) circuit which is represented in Figure 1. The first step of our analysis was to determine the frequency response by computing the transfer function of the whole circuit, followed by the determination of the input and output impedances.

In the last report we explained how the theoretical model of the transistor lacked the complexity needed to yield closer results to the simulation. In this assignment we dealt again with 2 transistors inside the OP-AMP. This alone means that the theoretical model used to analyse the circuit can differ significantly since it isn't expected to take into account the non-linearity of transistors. Adding to that is the complexity of the OP-AMP model used in Ngspice, especially the use of various capacitors and diodes which weren't taken into account in the octave analysis. Furthermore, another thing that was not taken into account was the parasitic capacitance of the 2 ransistors themselves, inside de OP-AMP. All these factors combined can cause discrepancies between the theoretical and simulation analysis.

That being said, almost every set of data value predicted in the theoretical section is matched in the simulation, with the only exception being the lower and upper cut-off frequency values, that caused our central frequency value to deviate a little. The bandwidth, despite still being approximately centered around the 1 KHz frequency, was much bigger (double the width). We've seen in the previous report that this might have been a source of error because we were not able to get a value for the upper cut off frequency. We believe that the issue with the bandwith this time around might have had to do with the parasitic capacitance of the transistors as a source. Another factor, which caused differences in the phase frequency response plots, is the two poles introduced by the OP-AMP itself that weren't considered in the simplified theoretical model. For theoretical purpose, the transfer function for the circuit only has two poles (one by the high pass, other by the low pass), and each one of the poles symbolizes (is directly correlated) with one of the cut-off frequencies. If in reality we have 4, not 2, poles, this can not be the case, so obviously the cut-off frequencies will be different, even if the central frequency mantains itself around the 1 kHz zone, which is the most important thing since it is the set of data being evaluated here, not the bandwidth.

Despite all of this, the merit figures are somewhat similar, especially if we consider the significant differences from previous assignments. Finally, we were this way able to design a OP-AMP band-pass filter which operates for a central frequency in the neighbourhood of the desired 1 kHz, with a gain of 40 dB as pretended. Despite the higher cost of the components used (aside from the OP-AMP sub-circuit itself), it allowed us to achieve better and more precise results, and even improving the figure of merit, so we do consider this laboratory assignment to be a success.
