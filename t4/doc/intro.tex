\section{Introduction}
\label{sec:introduction}
% state the learning objective 
The objective of this laboratory assignment is to build an audio amplifier circuit. 
For this laboratory assignment there is a figure of merit, depending on the results obtained in the NGSpice simulation. This figure takes into account the cost of the components used, and the results they provide - a desirably low cut off frequency, the voltage gain and usable bandwidth. The figure is calculated using the formula given by equation~\ref{eqn:merit}. The cost englobes the cost of transistors (0.1 units per transistor), resistors (1 unit per KOhm) and capacitors (1 unit per uF). The objective is to achieve the highest merit, so we tried different configurations of components data until we achieved our greatest figure of merit.
Since the human ear can perceive frequencies between 20 Hz to 20 kHz, the circuit was designed to amplify that frequency band.

\begin{equation}
\label{eqn:merit}
M=\frac{Gain*bandwidth}{cost*CutOffFrequency}
\end{equation}

\begin{figure}[h] \centering
\includegraphics[width=0.9\linewidth]{circuit.pdf}
\caption{Geometry of our Audio Amplifier circuit.}
\end{figure}

To analyse this circuit theoretically we will separate the circuit in two stages: the gain and output stage. We will compare these results with the ones obtained in the NGSpice simulation.
That being said, in Section 2 we present the theoretical models and calculations used to determine the operating point, gain, impedances and frequency response, in Section 3 we introduce the results obtained in the simulation. Finally, in Section 4 we compare the two set of results, looking for possible discrepancies and we lay out our conclusions.

In Table~\ref{tab:data}, we list the numeric values of the components used (nomenclatures consistent with the theoretical lecture).

\begin{table}[h]
  \centering
  \begin{tabular}{|l|r|}
    \hline    
    {\bf Name} & {\bf Values} \\ \hline
    \input{initialdata_tab} 
  \end{tabular}
  \caption{Values of components used in our analysis and simulation.}
  \label{tab:data}
\end{table}
\pagebreak
