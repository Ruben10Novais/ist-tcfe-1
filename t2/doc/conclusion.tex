\section{Conclusion}
\label{sec:conclusion}

In this laboratory assignment the objective of studying the presented circuit has been achieved. All the calculations for the Mesh and Node methods have been carried out using the Octave Maths tool and the circuit simulation has been done using the Ngspice tool. In a real, presential lab class, all sorts of experimental errors would be present: the resistance of the wiring, internal resistance of the sources, the temperature, external noise, etc. Besides that, reading and assembly errors would be expected. In this case, the simulation results matched the theoretical results: the reason for this match is the fact that this is a straightforward circuit containing only linear components, so if done correctly, the theoretical and simulation models cannot have noticeable differences. For more complex components, the theoretical and simulation models could differ greatly but this is not the case in this work. Despite the proximity of the values, there is still some difference between them (around the 6th decimal place). This can be considered negligible for the experiment in case, which doesn’t require a higher degree of accuracy since the whole purpose is just to study a simple circuit (it could be a problem if this belonged to a bigger functional structure where every small discrepancy between the expected and real data mattered). Nevertheless, these differences may have been caused by the employment of different numerical methods for solving the linear system and for different rounding criteria employed by the two different tools.
